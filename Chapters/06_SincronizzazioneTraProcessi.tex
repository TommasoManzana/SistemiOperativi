\chapter{Sincronizzazione tra processi}
Il modello astratto della sincronizzazione tra processi \`e quello del produttore-consumatore: il primo produce un messaggio e il secondo lo consuma in esecuzione concorrente. Essendo il
buffer limitato ci sono dei vincoli: non si pu\`o aggiungere in buffer pieni e non si pu\`o consumare da buffer vuoti.
\subsection{Buffer: modello software}
In software il buffer \`e circolare di $N$ posizioni, con $in$ che indica la prima posizione libera e $out$ la prima posizione occupata. Il buffer \`e vuoto quando $in=out$ e pieno 
quando $out = (in + 1) \% n$. Per semplicit\`a si utilizza una variabile counter per indicare il numero di elementi nel buffer. Il produttore svolger\`a la sua operazione incrementando
il counter se e solo se il counter \`e minore di $N$ e il consumatore lo decrementa solo se \`e maggiore di $0$. Essendo le operazioni di incremento e decremento non atomiche (sono pi\`u
istruzioni assembly) non \`e noto l'ordine di interleaving tra i processi e possono nascere delle inconsistenze in quanto produttore e consumatore possono modificare \emph{counter}
contemporaneamente. Si rende necessario proteggere l'accesso alla sezione critica. 
\subsection{Sezione critica}
Si intende per sezione critica una porzione di codice in cui si accede ad una risorsa condivisa. La soluzione al suo problema deve rispettare:
\begin{itemize}
	\item Mutua esclusione: unicamente un processo alla volta pu\`o accedere alla sezione critica.
	\item Progresso: solo i processi che stanno per entrare nella sezione critica possono decidere chi entra e la decisione non pu\`o essere rimandata all'infinito.
	\item Attesa limitata: deve esistere un massimo numero di volte per cui un processo pu\`o aspettare di seguito.
\end{itemize}
Un generico processo che accede ad una risorsa condivisa ha una struttura che contiene un'operazione ripetuta contenente prima delle operazioni sulla sezione critica le regole di 
ingresso alla sezione critica, le operazioni su di essa, una sezione di uscita dalla stessa e successivamente le operazioni sulla sezione non critica.
\subsubsection{Soluzione}
Assumendo una sincronizzazione in ambiente globale con condivisione di celle di memoria, le soluzioni software riguardano aggiunta di codice alle applicazioni senza nessun supporto 
hardware o del sistema operativo, mentre le soluzioni hardware riguardano codice alle applicazioni con supporto hardware. 
\section{Soluzioni software}
\lstinputlisting[caption={PROCESS i}, language={C}]{Algoritmi/IPC-algoritmo1.c}
Questo algoritmo richiede stretta alternanza tra i processi: possono entrare nella zona critica unicamente alternativamente. Inoltre non rispetta il criterio del progresso in quanto
non esiste alcuna nozione di stato.
\subsection{Algoritmo 2}
\lstinputlisting[caption={PROCESS i}, language={C}]{Algoritmi/IPC-algoritmo2.c}
Risolve il problema dell'algoritmo 1 ma l'esecuzione in sequenza dell'istruzione \emph{flag[] = true} da parte dei due processi porta a deadlock. Invertendo le istruzioni della sezione
di entrata si viola la mutua esclusione in quanto entrambi i processi possono trovarsi nella sezione critica se eseguono in sequenza il \emph{while} prima di impostare la \emph{flag} a
\emph{true}
\subsection{Algoritmo 3}
\lstinputlisting[caption={PROCESS i}, language={C}]{Algoritmi/IPC-algoritmo3.c}
\`E la soluzione corretta in quanto entra il primo processo che esegue \emph{turn = j} oppure \emph{turn = i}
\subsubsection{Dimostrazione}
\paragraph{Mutua esclusione}
$P_i$ entra nella sezione critica se e solo se $flag[j] = false$ o $turn = i$. Se $P_i$ e $P_j$ sono entrambi nella sezione critica allora $flag[i]=flag[j]=true$, ma $P_i$ e $P_j$ non
possono aver superato entrambi il \emph{while} perch\`e $turn$ vale $i$ oppure $j$, pertanto solo uno dei due $P$ \`e entrato. 
\paragraph{Progresso e attesa limitata}
Se $P_j$ non \`e pronto per entrare nella sezione critica allora $flag[j]=false$ e $P_i$ pu\`o entrare. Se $P_j$ ha impostato $flag[j]=true$ e si trova nel \emph{while} allora $turn = i$
oppure $turn = j$. Se $turn = i$ $P_i$ entra nella sezione critica, se $turn = j$ vi entra $P_j$. In ogni caso quando $P_j$ esce dalla sezione critica imposta $flag[j]=false$ e 
quindi $P_i$ pu\`o entrare nella sezione critica e $P_i$ entra nella sezione critica al massimo dopo un'entrata di $P_j$. 
\subsection{Algoritmo del fornaio}
Risolve il problema con $N$ processi: ogni processo sceglie un numero (\emph{choosing[i] = l}), il numero pi\`u basso viene servito per primo e per situazioni di numero identico si
usa un confronto a due livelli (\emph{numero, i}). L'algoritmo \`e corretto. 
\lstinputlisting[caption={PROCESS i}, language={C}]{Algoritmi/IPC-algoritmo-fornaio.c}
\section{Soluzioni hardware}
Un modo hardware per risolvere il problema della sezione critica \`e di disabilitare gli interrupt mentre una variabile condivisa viene modificata. Da questo nasce un problema in quanto
se il test per l'accesso \`e lungo gli interrupt devono essere disabilitati per troppo tempo. Un'alternativa \`e che l'operazione per l'accesso alla risorsa deve occupare un unico ciclo
di istruzione non interrompibile, con istruzioni atomiche di \emph{Test-and-set} e \emph{swap}
\subsection{Test and Set}
\lstinputlisting[caption={Test and Set}, language={C}]{Algoritmi/Test_and_set.c}
Il valore di ritorno \`e il valore di \emph{var} a cui viene assegnato \emph{true}. Si passa a questa funzione un lock che permette il controllo della sezione critica a un processo alla 
volta in quanto passa solo il primo processo che arriva e trova \emph{lock = false}. Quando il processo termina le operazioni sulla sezione critica pone \emph{lock = false}.
\subsection{Swap}
\lstinputlisting[caption={Swap}, language={C}]{Algoritmi/Swap.c}
Lo \emph{swap} viene utilizzato sul \emph{lock} e una variabile locale, che permette l'accesso solo quando quella locale diventa \emph{false} in modo da eliminare la competizione sul
\emph{lock}. Si noti come \emph{TestAndSet} e \emph{Swap} non rispettano attesa limitata in quanto manca l'equivalente della variabile \emph{turn} e sono pertanto necessarie variabili
addizionali.
\subsection{Test and Set con attesa limitata}
\lstinputlisting[caption={PROCESS i}, language={C}]{Algoritmi/TestSetLimitata.c}
\subsection{Conclusioni}
I vantaggi delle soluzioni hardware sono la scalabilit\`a in quanto indipendenti dal numero di processi coinvolti e il fatto che l'estensione a $N$ sezioni critiche \`e immediato. 
Offrono per\`o maggiore complessit\`a al programmatore rispetto alle soluzioni software e serve busy waiting con spreco di CPU.
\section{Semafori}
Si nota come le soluzioni precedenti non sono banali da aggiungere a programmi e sono basate su busy waiting. I semafori offrono una soluzione generica che funziona sempre. Si intende
per semaforo una variabile intera $S$ a cui si accede attraverso due primitive atomiche: 
\begin{itemize}
	\item Signal: \emph{V(s)} che incrementa il valore di $S$ di $1$. 
	\item Wait: \emph{P(s)} che tenta di decrementare il valore di $S$ di $1$, se \`e $0$ non si pu\`o ed \`e necessario attendere.
\end{itemize}
Esistono i semafori in versione binaria (con $S$ $0$ o $1$) o generica (con $S$ a valori interi maggiori o uguali di $0$).
\subsection{Semafori binari} 
Si noti come i semafori binari abbiano lo stesso potere espressivo di quelli a valori interi.
\lstinputlisting[caption={Implementazione concettuale}, language={C}]{Algoritmi/SemBinCon.c}
\subsubsection{Con busy waiting}
\begin{multicols}{2}
	\lstinputlisting[caption={Semaforo binario}, language={C}]{Algoritmi/SemBinWBW-P.c}
	\columnbreak
	\lstinputlisting[caption={Semaforo binario}, language={C}]{Algoritmi/SemBinWBW-V.c}
\end{multicols}
\subsection{Semafori interi}
Il problema dei semafori interi \`e garantirne l'atomicit\`a. 
\lstinputlisting[caption={Implementazione concettuale}, language={C}]{Algoritmi/SemIntCon.c}
\subsubsection{Con busy waiting}
Sia \emph{bool mutex} un semaforo binario inizializzato a \emph{TRUE} e \emph{bool delay} un semaforo binario inizializzato a \emph{FALSE}. Sia in \emph{P} che in \emph{V} l'operazione
\emph{P(mutex)} protegge $S$ da un'altra modifica. In \emph{V} \emph{V(delay)} permette la liberazione di un processo in attesa. In \emph{P} se qualcuno occupa il semaforo si attende, 
altrimenti lo si passa. 
\begin{multicols}{2}
	\lstinputlisting[caption={Semaforo intero}, language={C}]{Algoritmi/SemIntWBW-P.c}
	\columnbreak
	\lstinputlisting[caption={Semaforo intero}, language={C}]{Algoritmi/SemIntWBW-V.c}
\end{multicols}
\subsubsection{Senza busy waiting}
Nei semafori interi senza busy waiting si necessita di \emph{bool mutex}, un semaforo binario inizializzato a \emph{TRUE}. Inoltre il semaforo non \`e pi\`u un intero ma una 
\emph{struct} contenente un valore intero (con semantica analoga a quello con busy wating) e una lista contenente i PCB (process control block). Inoltre l'operazione di \emph{sleep()}
mette il processo nello stato di waiting mentre \emph{wakeup} lo mette nello stato ready. Si noti come si deve decidere l'ordine di \emph{wakeup} dei processi.
\lstinputlisting[caption={Semaforo intero}, language={C}]{Algoritmi/SemIntWoutBW.c}
\begin{multicols}{2}
	\lstinputlisting[caption={Semaforo intero}, language={C}]{Algoritmi/SemIntWoutBW-P.c}
	\columnbreak
	\lstinputlisting[caption={Semaforo intero}, language={C}]{Algoritmi/SemIntWoutBW-V.c}
\end{multicols}
In questo caso il busy waiting viene eliminato dalla entry section ma rimane nella \emph{P} e \emph{V} del mutex che essendo veloce porta a poco spreco di CPU. Un alternativa \`e 
disabilitare gli interrupt durante \emph{P} e \emph{V} in cui istruzioni di processi diversi non possono essere eseguite in modo alternato. 
\subsection{Implementazione}
Si noti come l'implementazione reale \`e diversa da quella concettuale: il valore di \emph{s} pu\`o diventare minore di zero per i semafori interi in quanto conta quanti processi ci 
sono in attesa. La lista dei PCB pu\`o essere FIFO (strong semaphore) garantendo cos\`i attesa limitata. Nella modalit\`a di implementazione con busy waiting (spinlock) la CPU controlla
attivamente il verificarsi della condizione di accesso alla sezione critica. \`E una soluzione scalabile e veloce, CPU-intensive e adatta per attese brevi, come l'accesso a memoria.
Nella modalit\`a senza busy waiting con sleep (mutex, semaforo) il processo viene messo inattesa che si verifichi la condizione di accesso alla sezione critica. \`E pi\`u lento e 
adatto per attese lunghe come l'I/O.
\subsection{Applicazioni}
Un semaforo binario con valore iniziale $1$ (mutex) viene utilizzato per la protezione di una sezione critica per $n$ processi. Un semaforo binario con valore iniziale $0$ viene 
utilizzato per la sincronizzazione del tipo di attesa di evento tra processi.
\subsubsection{Sezione critica}
Si dice mutex un semaforo binario di mutua esclusione. In questo caso $n$ processi condividono la variabile $s$.
\lstinputlisting[caption={Mutex}, language={C}]{Algoritmi/SemBinSezCrit.c}
\subsubsection{Semafori per attesa evento}
Si consideri il caso di due processi $P1$ e $P2$ che devono sincronizzarsi rispetto all'esecuzione di due operazioni $A$ e $B$ in modo che $P2$ possa eseguire $B$ solo dopo che $P1$
ha eseguito $A$. In questo caso si usa un semaforo binario $s$ inizializzato a $0$ tale che $P1$ svolga \emph{V(s)} dopo $A$ e $P2$ svolga $B$ dopo \emph{P(s)}.\\
Si consideri il caso di due processi $P1$ e $P2$ che devono sincronizzarsi rispetto all'esecuzione di un'operazione $A$ in maniera alternata. Vengono pertanto utilizzati due semafori 
$s1$ inizializzato a $1$ e $s2$ inizializzato a $0$ in modo che $P1$ svolga \emph{P(s1)} prima di $A$ e \emph{V(s2)} dopo, mentre $P2$ \emph{P(s2)} prima e \emph{V(s1)} dopo. 
\subsection{Limitazioni}
I semafori possono generare deadlock, in cui il processo viene bloccato in attesa di un evento che solo lui pu\`o generare o starvation, in cui avviene un'attesa indefinita all'interno
del semaforo. 
\section{Problemi classici dei semafori}
\subsection{Produttore consumatore}
Il problema del produttore consumatore consiste di due processi con accesso sullo stesso buffer di dimensione limitata. Il produttore scrive nel buffer  mentre il consumatore legge e
lo svuota. Il problema richiede tre semafori:
\begin{itemize}
	\item Mutex, un semaforo binario inizializzato a \emph{TRUE} che garantisce la mutua esclusione per il buffer.
	\item Empty, un semaforo intero inizializzato a $N$ che blocca $P$ se il buffer \`e pieno.
	\item Full, un semaforo intero inizializzato a $0$ che blocca $C$ se il buffer \`e vuoto.
\end{itemize}
\newpage
\begin{multicols}{2}
	\lstinputlisting[caption={Produttore}, language={C}]{Algoritmi/Produttore.c}
	\columnbreak
	\lstinputlisting[caption={Consumatore}, language={C}]{Algoritmi/Consumatore.c}
\end{multicols}
\subsection{Dining philosophers}
Si considerino $N$ filosofi che passano la vita mangiando e pensando. Si trova $1$ tavola con $N$ bacchette e una ciotola di riso. Se un filosofo pensa non interagisce con gli altri.
Se un filosofo ha fame prende $2$ bacchette e inizia a mangiare. Il filosofo pu\`o prendere solo le bacchette che sono alla sua destra e alla sua sinistra e pu\`o prenderne solo una 
alla volta. Se non ci sono due bacchette libere non pu\`o mangiare. Quando un filosofo termina di mangiare rilascia le bacchette.
\subsubsection{Prima soluzione}
\begin{multicols}{2}
	\paragraph{Dati condivisi}
	\begin{itemize}
		\item Ci sono $n$ semafori $s[N]$ inizializzati a $1$.
		\item $P(s[j])$ vuol dire che si cerca di prendere la bacchetta $j$;
		\item $V(s[j])$ rilascia la bacchetta $j$>
	\end{itemize}
	\paragraph{La soluzione \`e incompleta} C'\`e un possibile deadlock se tutti i filosofi tentano di prendere la bacchetta alla loro destra (sinistra) contemporaneamente. 
	\columnbreak
	\lstinputlisting[caption={Dining philosopher}, language={C}]{Algoritmi/Philosopher1.c}
\end{multicols}
\paragraph{Deadlock}
Questa soluzione presenta un problema in quanto nasce un deadlock nel caso in cui ogni filosofo prende la bacchetta di destra e si trovano tutti ad aspettare che si liberi quella di 
sinistra che nessuno pu\`o rilasciare. 
\paragraph{Possibili soluzioni parziali}
\begin{itemize}
	\item Si permette solo a quattro filosofi di mangiare contemporaneamente.
	\item Soluzione asimmetrica in cui i filosofi in posizione pari prendono la bacchetta sinistra seguita dalla destra mentre i filosofi in posizione dispari fanno il contrario.
	\item I filosofi si passano un token.
	\item Si permette ai filosofi di prendere la bacchetta solo se sono entrambe disponibili.
\end{itemize}
\subsubsection{Soluzione corretta}
In questa soluzione ogni filosofo si pu\`o trovare in tre stati: pensante (\emph{THINKING}), affamato (\emph{HUNGRY}) e  mangiante (\emph{EATING}). Le variabili condivise sono un 
semaforo mutex inizializzato a $1$ $N$ semafori $f[N] = 0$ e lo stato di ogni filosofo $stato[N]$. 
\begin{multicols}{2}
	\lstinputlisting[caption={Dining philosopher}, language={C}]{Algoritmi/Philosopher2.c}
	\lstinputlisting[caption={Test}, language={C}]{Algoritmi/Philosopher2-test.c}
	\columnbreak	
	\lstinputlisting[caption={Drop fork}, language={C}]{Algoritmi/Philosopher2-dropfork.c}
	\lstinputlisting[caption={Take fork}, language={C}]{Algoritmi/Philosopher2-takefork.c}
\end{multicols}
\subsection{Sleepy barber}
Un negozio ha una sala d'attesa con $N$ sedie ed una stanza con la sedia del barbiere. In assenza di clienti il barbiere si addormenta. Quando entra un cliente se le sedie sono 
occupate il cliente se ne va, se il barbiere \`e occupato si siede e se \`e addormentato lo sveglia. 
\subsubsection{Soluzione}
\paragraph{Dati condivisi}
\begin{itemize}
	\item Semaforo intero customer inizializzato a zero che sveglia il barbiere.
	\item Semaforo binario barbers inizializzato a zero che rappresenta lo stato del barbiere.
	\item Semaforo binario mutex inizializzato a $1$ che protegge la sezione critica.
	\item \emph{int waiting = 0} che conta i clienti in attesa.
\end{itemize}
\begin{multicols}{2}
	\lstinputlisting[caption={Barber}, language={C}]{Algoritmi/Barber.c}
	\columnbreak
	\lstinputlisting[caption={Consumer}, language={C}]{Algoritmi/Customer.c}
\end{multicols}
\subsection{Limitazioni dei semafori}
L'utilizzo dei semafori presenta delle difficolt\`a in quanto risulta difficile scrivere programmi e la correttezza delle soluzioni \`e difficilmente dimostrabile. In alternativa 
vengono pertanto utilizzati specifici costrutti forniti da linguaggi di programmazione ad alto livello come monitor, classi synchronized in Java e CCR (conditional critical region) e 
altri. 
\section{Monitor}
Sono costrutti per la condivisione sicura ed efficiente di dati da processi. Sono simili al concetto di classe.
\lstinputlisting[caption={Monitor}, language={C}]{Algoritmi/Monitor.c}
Le variabili del monitor sono visibili solo all'interno del monitor stesso e le procedure del monitor accedono solo alle variabili definite nel monitor. Un solo processo alla volta 
risulta attivo in un monitor in modo che il programmatore non debba codificare esplicitamente la mutua esclusione. 
\subsection{Operazioni del monitor}
Per permettere ad un processo di attendere all'interno del monitor si rendono necessari opportune sincronizzazioni: le variabili condition dichiarate all'interno del monitor e 
accessibili solo tramite due primitive analoghe a quelle dei semafori: \emph{wait()} come \emph{P()} e \emph{signal()} come \emph{V()}. Il processo che invoca \emph{x.wait()} \`e 
bloccato fino all'invocazione della corrispondente \emph{x.signal()} da parte di un altro. 
\subsubsection{Wait}
La \emph{wait} blocca sempre il processo che la chiama e si deve pertanto prestare attenzione alla logica che regola tale chiamata. 
\subsubsection{Signal}
La \emph{signal} sveglia esattamente un processo e se ne trovano pi\`u in attesa lo scheduler decide quale processo pu\`o entrare. Se nessun processo si trova in attesa non c'\`e nessun
effetto. Successivamente a questa operazione il processo che la invoca si pu\`o bloccare passando l'esecuzione al processo sbloccato o esce dal monito, caso in cui signal deve essere
l'ultima istruzione di una procedura. 
\subsubsection{Buffer produttore consumatore}
\begin{multicols}{2}
	\lstinputlisting[caption={Produttore}, language={C}]{Algoritmi/MonitorProduttore.c}
	\columnbreak
	\lstinputlisting[caption={Consumatore}, language={C}]{Algoritmi/MonitorConsumatore.c}
\end{multicols}
\lstinputlisting[caption={Monitor per buffer}, language={C}]{Algoritmi/MonitorProducerConsumer.c}
\subsubsection{Semaforo binario nel monitor}
\lstinputlisting[caption={Semaforo binario nel monitor}, language={C}]{Algoritmi/MonitorSemBin.c}
\subsection{Limitazioni}
Pur essendo meno proni ad errori rispetto ai semafori i monitor sono forniti da pochi linguaggi e richiedono sempre presenza di memoria condivisa. 
\section{Sincronizzazione in Java}
La sezione critica viene indicata dalla keyword synchronized e i metodi synchronized possono essere eseguiti da un solo thread alla volta e vengono realizzati mantenendo un singolo lock
detto monitor per oggetto. I metodi static synchronized presentano un solo lock per classe, mentre per i blocchi synchronized \`e possibile mettere un lock su qualsiasi oggetto per 
definire una sezione critica. Altri metodi di sincronizzazione sono \emph{wait()} \emph{notify()} e \emph{notifyAll()}, che vengono ereditati da tutti gli oggetti. 
\subsection{Buffer produttore consumatore}
\begin{multicols}{2}
	\lstinputlisting[caption={Bounded buffer}, language={Java}]{Algoritmi/BoundedBuffer.java}
	\columnbreak
	\lstinputlisting[caption={Deposit}, language={Java}]{Algoritmi/Deposit.java}
	\lstinputlisting[caption={Remove}, language={Java}]{Algoritmi/Remove.java}
\end{multicols}
\section{Conclusioni}
Il problema della soluzione critica \`e un'astrazione della concorrenza tra processi. Esistono soluzioni con diversi compromessi tra complessit\`a e difficolt\`a di utilizzo. Si 
deve prestare attenzione alla gestione del blocco critico di un insieme di processi (deadlock) dipendente dalla sequenza temporale degli accessi.
\section{Problema degli scrittori e dei lettori}
C'\`e un area di dati condivisa ed esistono dei lettori che possono solo leggere i dati e scrittori che possono solo scriverli. Pi\`u lettori possono leggere il file contemporaneamente, 
solo uno scrittore alla volta pu\`o scrivere nel file, se uno scrittore sta scrivendo nel file nessun lettore pu\`o leggerlo, gli scrittori non possono leggere e i lettori non possono
essere anche scrittori. 
\lstinputlisting[caption={Scrittore}, language={C}]{Algoritmi/Scrittore.c}
\lstinputlisting[caption={Lettore}, language={C}]{Algoritmi/Lettore.c}
