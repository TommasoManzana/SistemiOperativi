\chapter{Gestione della memoria secondaria}
\section{Tipologie di supporto}
\subsection{Nastri magnetici}
I nastri magnetici sono formati da una sottile striscia di materiale plastico rivestita di un materiale magnetizzabile. Furono usati per memorizzare dati digitali per la prima volta
nel $1951$. La massima capienza \`e di circa $5TB$ nel $2011$. Sono ad accesso sequenziale e molto pi\`u lenti della memoria principale e dei dischi rigidi in termini di tempo di 
accesso. Il riposizionamento della testina di lettura richiede decine di secondi. Sono stati rimpiazzati da dischi magnetici e memorie a stato solido e usati solo per backup. 
\subsection{Dischi magnetici}
Sono piatti di alluminio o di altro materiale ricoperti di materiale ferromagnetico con una massima capienza di $4TB$ nel $2011$. Il fattore di forma o diametro \`e sempre pi\`u piccolo
in modo da raggiungere maggiori velocit\`a di rotazione. Parte da $3.5$ pollici per i sistemi desktop e fino a $1$ pollice per i mobili. La lettura e la scrittura avvengono tramite una
testina sospesa sulla superficie magnetica. Nella scrittura il passaggio di corrente positiva o negativa attraverso la testina magnetizza la superficie, mentre nella lettura il 
passaggio sopra un'area magnetizzata induce una corrente positiva o negativa nella testina. 
\subsubsection{Struttura di un disco}
Da un punto di vista fisico il disco \`e costituito da superfici (platter, i dischi), cilindri (cylinder, le tracce di tutti i dischi con la stessa posizione), tracce (cerchi di settori
all'interno di un disco) e settori (sottoaree del disco disposte in cerchi concentrici). 
\paragraph{Settore}
Il settore \`e l'unita pi\`u piccola di informazione che pu\`o essere letta o scritta su disco. Ha una dimensione variabile tra i $32$ byte e i $4KB$ (tipicamente di $512$ byte). Sono 
generalmente raggruppati logicamente in cluster per aumentare l'efficienza. La dimensione dei cluster dipende dalla grandezza del disco e dal sistema operativo (da $2KB$ a $32KB$). Un
file occupa sempre almeno un cluster. Per accedere a un settore si necessita di sapere la superficie, la traccia e il settore stesso. 
\subsubsection{Tempo di accesso al disco}
Il tempo di accesso \`e la somma del seek time (tempo necessario a spostare la testina sulla traccia), del latency time (latenza, il tempo necessario a posizionare il settore desiderato 
sotto la testina che dipende dalla velocit\`a di rotazione) e del transfer time (il tempo necessario al settore per passare sotto la testina). Il seek time va da $3ms$ fino a $15ms$, 
mediamente $9ms$ per dischi desktop. La latenza, considerando in media mezza rotazione del disco per ogni accesso \`e $0.5\cdot\bigr(\frac{60}{\text{velcit\`a di rotazione}}\bigl)$. Per
un disco da $7200 rpm$ \`e di $4.16ms$. Il transfer time \`e il rapporto tra la dimensione del blocco e la velocit\`a di trasferimento. Per un disco da $7200rpm$ si ha: disk-to-buffer
$1030\frac{Mbits}{sec}$ e buffer-to-computer $300\frac{MB}{sec}$. Si noti come il seek time \`e dominante e pertanto dato il grande numero di processi che accedono al disco \`e 
necessario minimizzare il seek time tramite algoritmi di scheduling per ordinare gli accessi al disco in modo da minimizzare il tempo di accesso totale (riducendo lo spostamento della
testina) o massimizzare la banda, il numero di byte trasferiti nell'unit\`a di tempo, che misura la velocit\`a effettiva. 
\subsection{Dispositivi a stato solido}
Utilizzano chip NVRAM o memorie flash NAND per memorizzare dati in modo non volatile. Usano la stessa interfaccia dei dischi fissi e pertanto possono rimpiazzarli facilmente. Hanno una
capacit\`a fino a $2TB$. 
\subsubsection{Performance delle flash memory}
Hanno letture simili a DRAM negli ordini dei nanosecondi e scritture simili ai dischi magnetici nell'ordine dei millisecondi. Rispetto ai dischi fissi sono meno soggette a danni, ma
con un limite superiore sul numero di write, pi\`u silenziosi, pi\`u efficienti nel tempo di accesso, non necessitano di essere deframmentate e sono pi\`u costose. 
\section{Scheduling degli accessi a disco}
Logicamente il disco pu\`o essere considerato come un vettore unidimensionale di blocchi logici. Un blocco o cluster \`e l'unit\`a minima di trasferimento. Il vettore \`e mappato 
sequenzialmente sui settori del disco. Il settore $0$ \`e il primo settore della prima traccia del cilindro pi\`u esterno e la numerazione procede gerarchicamente per settore, tracce e
piatti. 
\subsection{Disk scheduling}
Un processo che necessita I/O esegue una system call e il sistema operativo usa la vista logica del disco e una sequenza di accessi sono una sequenza di indici del vettore. Nelle 
richieste sono anche contenute informazioni come il tipo di accesso (R/W), l'indirizzo di memoria destinazione e la quantit\`a di dati da trasferire. 
\subsection{Algoritmi di disk scheduling}
In questi algoritmi bisogna sempre tenere conto del compromesso tra costo ed efficacia. Sono valutati tramite una sequenza di accessi. 
\subsubsection{First-come-first-served (FCFS)}
In questo algoritmo le richieste sono processate nell'ordine di arrivo. 
\subsubsection{Shortest seek time first (SSTF)}
In questo algoritmo si fa la scelta pi\`u efficiente localmente: si seleziona la richiesta con il minimo spostamento rispetto alla posizione attuale della testina. Simile al SJF con
possibilit\`a di starvation di alcune richieste. Non \`e ottimo. 
\subsubsection{SCAN}
La natura dinamica delle richieste porta all'algoritmo SCAN in cui la testina parte da un'estremit\`a del disco, si sposta verso l'altra estremit\`a servendo tutte le richieste 
correnti. Arrivato all'altra estremit\`a riparte nella direzione opposta servendo le nuove richieste. Viene detto algoritmo dell'ascensore. 
\subsubsection{SCAN circolare (CSCAN)}
Come SCAN, ma quando la testina arriva ad un'estremit\`a riparte immediatamente da $0$ senza servire altre richieste. Il disco viene considerato come lista circolare e il tempo di 
attesa \`e pi\`u uniforme rispetto a SCAN. Alla fine di una scansione ci saranno pi\`u richieste all'altro estremo. 
\subsubsection{(C-)LOOK, variante dello SCAN}
La testina non arriva fino all'estremit\`a del disco. Cambia direzione (LOOK) o riparte dalla prima traccia (C-LOOK) non appena non ci sono pi\`u richieste in quella direzione.
\subsubsection{N-step SCAN, variante dello SCAN}
Per evitare che la testina rimanga sempre nella stessa zona la coda delle richieste viene partizionata in pi\`u code di dimensione massima $N$. Quando una coda viene processata per
il servizio gli accessi in arrivo riempiono altre code. Dopo che una coda \`e stata riempita non \`e possibile riordinare le richieste. Le code sature vengono servite nello scan 
successivo. Per $N$ grandi degenera in SCAN, mentre per $N = 1$ degenera in FCFS. FSCAN \`e simile ma con due sole code. 
\subsubsection{Last-in-last-out (LIFO)}
In certi casi pu\`o essere utile schedulare gli accessi in base all'ordine inverso di arrivo. \`E utile nel caso di accessi con elevata localit\`a ma offre possibilit\`a di 
starvation. 
\subsubsection{Analisi degli algoritmi}
Nessuno degli algoritmi considerati \`e ottimo in quanto quello ottimo sarebbe poco efficiente. L'analisi \`e influenzata dalla distribuzione di numero e dimensione degli accessi in
quanto pi\`u l'accesso \`e grande pi\`u il peso relativo del seek time diminuisce e dall'organizzazione delle informazioni su disco (il file system) e l'accesso alla directory 
essenziale tipicamente lontane dalle estremit\`a del disco. In generale SCAN e C-SCAN sono migliori per sistemi con molti accessi a disco. Il disk-scheduling viene spesso implementato
come modulo indipendente dal sistema operativo con diverse scelte di algoritmo. 
\section{Gestione del disco}
\subsection{Formattazione dei dischi}
Nella formattazione di basso livello o fisica il disco viene diviso in settori che il controllore pu\`o leggere o scrivere. Per usare un disco come contenitore di file il sistema 
operativo deve memorizzare le proprie strutture del disco che viene partizionato in uno o pi\`u gruppi di cilindri detti partizioni. Avviene una formattazione logica per creare un 
file system e il programma di boot per inizializzare il sistema. Il programma di bootstrap memorizzato nella ROM (bootstrap loader) carica il bootstrap dal disco (dai blocchi di root), 
carica i driver dei dispositivi e lancia l'avvio del sistema operativo. Nella formattazione fisica si suddivide il disco in settori, si identificano e si aggiunge lo spazio per la 
correzione degli errori (ECC). Nella formattazione logica si inizializza il file system, la lista di spazio occupato e libero e le directory vuote.
\subsection{Gestione dei blocchi difettosi}
L'ECC (error correction code) serve a capire in lettura o scrittore se un settore contiene dati corretti o meno. Per la lettura il controllore legge un settore insieme all'ECC e calcola
l'ECC per i dati appena letti. Se il risultato \`e diverso si trova un errore (bad block). L'errore va segnalato e i bad block si possono gestire offline o online. 
\subsubsection{Gestione offline dei bad block}
Durante la formattazione logica si individuano i bad block e si mettono una lista, si rimuovono e si marca l'entry nella FAT. Successivamente si pu\`o eseguire un'utility per isolare i 
bad block come. Questo algoritmo \`e tipico dei controllori IDE. 
\subsubsection{Gestione online dei bad block}
Il controllore mappa il bad block su un blocco buono non usato. Si deve pertanto disporre di blocchi di scorta riservati (sector sparing). Quando il sistema operativo richiede il bad 
block il controllore mappa in modo trasparente l'accesso al bad block sul blocco di scorta. Potrebbe inficiare le ottimizzatone fornite dallo scheduling se non si allocassero spare
block in ogni cilindro. 
\subsection{Gestione dello spazio di swap}
Lo spazio di swap viene usato dalla memoria virtuale come estensione della RAM. Si pu\`o ricavare dal file system attraverso comuni primitive di accesso ai file, ma \`e inefficiente in
quanto esiste un costo addizionale per l'attraversamento delle strutture dati per le directory e i file. Si pu\`o porre in una partizione separata che non contiene informazioni e 
strutture relative al file system e usa uno speciale gestore detto swap daemon. Lo spazio di swap pu\`o essere allocato quando il processo viene creato e viene assegnato spazio per 
pagine di istruzioni e di dati. Il kernel usa due mappe di swap per tracciare l'uso dello spazio di swap. Oppure pu\`o essere allocato quando una pagina viene forzata fuori dalla memoria
fisica. 
